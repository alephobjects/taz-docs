%
% Advanced.tex
% Advanced Techniques
%
% LulzBot TAZ User Manual
%
% Copyright (C) 2014 Aleph Objects, Inc.
%
% This document is licensed under the Creative Commons Attribution 4.0
% International Public License (CC BY-SA 4.0) by Aleph Objects, Inc.
%
%

\section{Intro}
\index{advanced techniques}
\glossary{3D Printer}{Also referred to as additive manufacturing, is the process of fabricating objects from 3D model data, through the deposition of a material in accumulative layers.}
After you become familiar with printing using the default settings, a few advanced techniques may help in getting better and more consistent prints from the TAZ 3D printer. Some of these instructions are items and materials not included with the TAZ. With any of these additional items or materials, follow safety and usage guidelines as instructed by the manufacturer.

\section{Changing nozzles}
\index{nozzle}
\glossary{Nozzle}{The metal tip at the bottom of the hot end. It has a small hole where the plastic filament comes out of the printer.}
\index{high resolution}
The TAZ 3D printer ships with a standard 0.5mm nozzle which allows small-layer resolution and up to 0.5mm layers. Although the 0.5mm nozzle will be perfect for most printing applications, LulzBot may also offer smaller or larger nozzle sizes.
\glossary{Layer height}{The thickness of each individual deposited layer of the three-dimensional model when cut with a slicing program.}


\index{hot end}
\glossary{Hot end}{The hot end is the whole part where the plastic melts, including the nozzle, heater block, thermistor, and heat sink. The LulzBot\textsuperscript{\miniscule{\texttrademark}} Hexagon all-metal hot end comes standard on the TAZ 5.}
\index{threaded extension}
\glossary{Threaded extension}{Used to separate the heater block and nozzle from the PEEK insulator. The plastic filament passes through the threaded extension into the melting chamber.}
%\glossary{PEEK}{Polyether ether ketone: an organic polymer used to insulate the hot end due to its mechanical properties at elevated temperatures.}
%In most cases the nozzle is best changed when the hot end is slightly warm. NEVER try to remove the nozzle when the hot end is at extrusion temperature. At higher temperatures the threaded extension expands in the nozzle causing the nozzle to bind if turned. Heat the hot end to \texttt{160°C}. This will soften the plastic inside the hot end and allow the nozzle to be loosened off the threaded extension. Power off the printer and before it cools unscrew the nozzle. Take care when removing the nozzle while the hot end is hot. Wear leather gloves or use a towel to turn the nozzle off the hot end.

\index{wrench}
\index{heater block}
\glossary{Heater block}{Machined from aluminum, the heater block generates heat with a heater resistor and uses a thermistor to measure the temperature.}
\glossary{Thermistor}{A special type of resistor that changes resistance based on temperature. It is used to measure temperature on the nozzle and the heated bed.}
\glossary{Heater resistor}{A special type of resistor that is used to apply heat in a small area.}
To change the nozzle you will need the included 13mm wrench and needle nose pliers. Use the pliers to hold the rectangular aluminum heater block away from the heater resistor and thermistor wires. To minimize risk to the hot end wiring, from the rear of the printer hold onto the heater block.

Using the 13mm wrench, turn the nozzle counter-clock-wise. Make sure the nozzle is turning off of the threaded aluminum extension that runs up through the heater block. Do not allow the heater block to turn. This can put strain on, and possibly damage, the wiring.

Once you have removed the nozzle you can then thread on the other nozzle size you would like to use. Make sure the nozzle has threaded correctly onto the threaded extension before trying to turn it with the wrench. Turn the nozzle clock-wise until it tightens against the heater block.

After installing the new nozzle you may need to adjust your Z home trigger setting before printing again. Refer to the "Quick Start" guides included with your TAZ 3D printer or rotate the Z axis endstop trigger counter-clockwise to start with a higher Z axis homing height.

%\index{anti-seize}
%If you will be changing nozzles frequently we suggest reapplying a small amount of high temperature anti-seize compound to the inside threads of the nozzles. You will need an anti-seize capable of temperatures of at least \texttt{250°C}.

\section{ABS/Acetone Glue}
\label{sec:ABS/Acetone Glue}
\index{acetone}
\glossary{Acetone}{A colorless, volatile, flammable liquid ketone, (CH3)2CO, used as a solvent for ABS.}
Acetone is not included or required with the TAZ 3D printer. An acetone safety label is included for the HDPE bottle.

\index{ABS}
\textcolor{red}{Acetone can cause skin irritation when prolonged skin contact occurs. We recommend using acetone safe gloves when applying the ABS/acetone glue. Use the ABS/acetone glue in a well-ventilated space. Leave the mixture bottle closed except when applying a small amount to the wiping towel. Acetone liquid and vapors are highly flammable. Keep acetone away from open flames and high temperature sources, including the 3D printer. Read the warnings label on your purchased acetone packaging for additional warnings.}

\index{warping}
\index{bottle}
You may find that during printing, printed parts lift off the print surface on the corners. If you are seeing this problem you can make an ABS/acetone glue to apply to the print surface. Using the HDPE acetone-safe bottle included in the printer kit, fill the bottle 3/4 full with acetone. Now cut eight 75-mm lengths of ABS filament and put them in the bottle with the acetone. Allow the ABS filament to dissolve for 2 hours.

%\index{PET sheet}
%\glossary{PET}{Polyethylene terephthalate.}
%To apply the acetone/ABS mixture, put a small amount onto a paper towel. Now, rub the towel onto the cool PET print surface to apply a \emph{thin} layer of ABS. Generally only one thin layer of the acetone/ABS solution is needed. However, you can apply multiple coats if needed.

\section{Using 1.75mm filament}
\index{1.75mm filament}
\index{PTFE tube}
\glossary{PTFE}{Polytetrafluoroethylene is a synthetic fluoropolymer used in the Budaschnozzle for it's low coefficient of friction. The TAZ 5 does not use a hot end with a PTFE insert.}

The TAZ 3D printer is set up to use 3mm plastic filament by default. The TAZ is only capable of printing 1.75mm filament by purchasing a compatible 1.75mm hot end.
