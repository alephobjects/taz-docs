%!TEX root = Slic3r-Manual.tex

\subsection{\texttt{Overview}} % (fold)
\label{sec:overview}

Slic3r is a tool which translates digital 3D models into instructions that are understood by a 3D printer.  It slices the model into horizontal layers and generates suitable paths to fill them.

Slic3r is already bundled with the many of the most well-known host software packages: Pronterface, Repetier-Host, ReplicatorG, and can be used as a standalone program.

This manual will provide guidance on how to install, configure and utilize Slic3r in order to produce excellent prints.

This portion of the manual is derived from the complete Slic3r manual. It has been customized for TAZ users. The original unabridged version can be found at \texttt{manual.slic3r.org}. Due to the introduction of Cura LulzBot\textsuperscript{\miniscule{\texttrademark}} Edition, the recommended printer host and slicer, this abriged Slic3r manual may not be current.

% subsection overview (end)


\subsection{\texttt{Goals \& Philosophy}} % (fold)
\label{sec:goals_philosophy}

Slic3r is an original project started in 2011 by Alessandro Ranellucci (aka. Sound), who used his considerable knowledge of the Perl language to create a fast and easy to use application.  Readability and maintainability of the code are among the design goals.

The program is under constant refinement, from Alessandro and the other contributors to the project, with new features and bug fixes being released on a regular basis.

% subsection goals_philosophy (end)


\subsection{\texttt{Donating}} % (fold)
\label{sec:donating}

Slic3r started as a one-man job, developed solely by Alessandro in his spare time, and as a freelance developer this has a direct cost for him.  By generously releasing Slic3r to the public as open source software, under the GPL license, he has enabled many to benefit from his work.

The opportunity to say thank you via a donation exists.  More details can be found at: \texttt{http://slic3r.org/donations}.

% subsection donating (end)
