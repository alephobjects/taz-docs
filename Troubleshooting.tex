%
% Troubleshooting.tex
%
% LulzBot TAZ User Manual
%
% Copyright (C) 2014 Aleph Objects, Inc.
%
% This document is licensed under the Creative Commons Attribution 4.0
% International Public License (CC BY-SA 4.0) by Aleph Objects, Inc.
%

\section{Troubleshooting}
\subsection{Error Codes}
\index{Error Codes}
The following error codes may be displayed within the Cura LulzBot Edition control window, or displayed on the Graphical LCD controller. 

\subsubsection{MINTEMP}
\index{MINTEMP}
The thermistor for one of the hot ends is reading 0°C or below. This is usually the result of a break in the thermistor wiring.

\subsubsection{MAXTEMP}
\index{MAXTEMP}
The thermistor reading has gone above the maximum temperature set in the firmware (300°C for the TAZ and Mini and most tool heads). This could be caused by shorted thermistor wires, a failing heater or poorly set PID values causing the hot end to overrun it’s set temp.

\subsubsection{Thermal Error E1}
\index{Thermal Error E1}
Triggered by the primary hot end dropping in temperature beyond a threshhold set in the firmware. It can be caused by a failing heater cartridge or excessive fan on the hot end, or in a worst case scenario, by the thermistor coming out of the tool head during printing.

\subsubsection{Thermal Error E2}
\index{Thermal Error E2}
Triggered by the secondary hot end dropping in temperature beyond a threshhold set in the firmware. It can be caused by a failing heater cartridge or excessive fan on the hot end, or in a worst case scenario, by the thermistor coming out of the tool head during printing.

\subsubsection{Thermal Error Bed}
\index{Thermal Error Bed}
This is triggered by the bed dropping in temperature beyond a threshhold set in the firmware. It can be caused by a failing bed heater or wiring or excessive fan on the bed, or in a worst case scenario, by the thermistor coming out of the bed during printing (unlikely).

\subsubsection{PROBE FAIL CLEAN NOZZLE}
\index{PROBE FAIL CLEAN NOZZLE}
This means that the printer has detected a failed wiping routine and has not been able to correct itself with 3 rewipe sequences. This is usually caused by the wiping temps not being correct for the material that’s coating the nozzle.
Resolution:
\begin{itemize}
\item Use the correct Quickprint profile in Cura LulzBot Edition.
\item If you’ve just switched from a material that is printed at a higher temperature to one that prints at a lower temperature, try heating the hot end to the wipe temperature recommended in that material’s Quickprint profile. 

A printed wiper pad handle can make this much easier: \texttt{http://devel.lulzbot.com/wiper\textunderscore pads/wiper-pad-handle}
\end{itemize}

\subsubsection{HEATING FAILED}
The hot end was not able to reach the target temperature.
