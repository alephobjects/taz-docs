\section{Read Me First!}
\index{warnings}
\index{hazards}
\textcolor{red}{READ THIS MANUAL COMPLETELY BEFORE UNPACKING AND POWERING UP YOUR PRINTER.}

\section{Hazards and Warnings}

The TAZ 3D printer has motorized and heated parts. When the printer is in operation always be aware of possible hazards.

\subsection{\textcolor{red}{Electric Shock Hazard}}
\index{electronics}
\index{wires}
\index{power supply}
Never open the electronics case when the printer is powered on. Before removing the electronics case cover always power down the printer and completely turn off and unplug the power supply and allow the power supply to discharge for at least 1 minute.

\subsection{\textcolor{red}{Burn Hazard}}
\index{extruder}
\index{heater block}
\index{temperature}
\index{burns}
Never touch the extruder nozzle or heater block without first turning off the hot end and allowing it to completely cool down. The hot end can take up to twenty minutes to completely cool. Never touch recently extruded plastic. The plastic can stick to your skin and cause burns. The heated bed can reach high temperatures capable of causing burns.

\subsection{\textcolor{red}{Fire Hazard}}
Never place flammable materials or liquids on or near the printer when powered or in operation. Liquid acetone and vapors are extremely flammable.

\subsection{\textcolor{red}{Pinch Hazard}}

When the printer is in operation take care to never put your fingers in the moving parts including the belts, pulleys, or gears. Tie back long hair or clothing that can get caught in the moving parts of the printer.

\subsection{\textcolor{red}{Static Charge}}
\index{static}
Make sure to ground yourself before touching the printer, especially the electronics. Electrostatic discharge can damage electronic components. Ground yourself touch a grounded source.

\subsection{\textcolor{red}{Age Warning}}

For users under the age of 18, adult supervision is recommended. Beware of choking hazards around small children.

