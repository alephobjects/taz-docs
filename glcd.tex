
The Graphic LCD allows you to print with the TAZ 3D printer without needing to have a connected computer or use a host software (Printrun). This will allow for more efficient space in the workspace and free up a computer for other tasks.

In the following sections you will find general information on using the Graphic LCD, how to transfer .gcode files to the included sd card, heat up the printer, start a print, and make configuration adjustments.

\section{Graphic LCD or Printrun Host?}
The Graphic LCD is perfect for normal day to day print use and will be used in the majority of your print jobs. However, there are some instances where you will want to use the Printrun software. A few examples of when you would want to plug the USB cable back in and fire up Printrun:
\begin{itemize}
	\item When performing calibration checks a number of manual movements are required. Because of this it is easier and faster to make the required manual movements within Printrun. Calibration checks can be done with the Graphic LCD, but require a number of repetitive menu selections.
	\item Printrun offers a number of extra options for advanced users, including: custom gcode input, output display, and pronsole. Pronsole is the command line portion of Printrun which can be used in scripts for automation or controlling a printer through remote access (SSH for example).
\end{itemize}

\section{Multiple Connections}
Because the TAZ 3D printer can be controlled by the Graphic LCD and by host software, caution is advised when the USB is connected to the printer. A general rule is: once you have started a print with either the Graphic LCD or Printrun, for the rest of the print only use that controller. When printing with the Graphic LCD, never try to connect through USB in the Printrun host software; wait until the print is complete, and then connect in Printrun.

\section{Putting Print Files On the SD Card}
To print from the Graphic LCD you will need to transfer .gcode print files onto the SD card. Follow the normal steps, as explained in the Slic3r manual, to create .gcode print files on your computer. Insert the SD card into your computer using a SD reader slot or USB SD card reader. Open a file browser / manager and locate the created .gcode files; drag and drop or paste the .gcode files to the SD card. Once the files have transfered, eject the SD card from your computer and insert it back into the SD card slot on the left side of the Graphic LCD case.

\section{Printing with the Graphic LCD}
\subsection(Using the Selection Knob}
To navigate through the LCD menu use the selection knob by rotating to scroll through selections and pressing the knob to make a selection. From the main status screen, press the knob to move into the menu screen. To move backwards in the menu tree, select the top most menu selection on the current screen. Selections that will move you backwards through the menu tree are noted by an upwards facing arrow. Note that if the menu is left idle it will automatically move back to the main status screen.

\subsection{Preparing for a Print}
Before starting a print you will need to set the hot end and heat bed to the appropriate temperatures for the filament type you are using. To quickly set the printer to preheat, for ABS or PLA filament, click the selection knob to bring up the menu and select \texttt{Prepare}. Select \texttt{Preheat PLA} for PLA filament or \texttt{Preheat ABS} for ABS filament. This will set the temperatures for the hot end and heat bed and begin bringing the printer up to print ready temperature.

