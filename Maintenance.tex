\section{Overview}
\index{maintenance}
There is little maintenance needed in keeping your TAZ 3D printer running. Depending on your rate of use you will want to perform a quick check of your printer every 2-4 weeks. The following maintenance guide lines will keep your printer printing quality parts.

\section{Smooth Rods}
\index{smooth rods}
\index{bushings}
\index{lubricant}
Wipe the smooth steel rods with a clean rag or paper towel. The linear bushings leave a solid lubricant that builds up over time. If you begin hearing squeaking noises while the printer is printing, this is likely a sign that the smooth rods need to be cleaned. NOTE: never apply any lubricant or cleaning agent to the smooth rods; the bushings are self lubricating.

\section{PET Sheets}
\index{PET sheet}
\index{glass}
\index{acetone}
After repeated use, the PET sheet print surface will begin to wear. Replacement PET sheets are available on \texttt{LulzBot.com}. To replace the PET print surface, peel off the worn PET sheet from the glass print surface. If there is any glue or plastic residue left on the glass surface, clean it with acetone or an alcohol based glass cleaner. Peel a corner of the clear plastic away from the green PET sheet and apply the corner of the PET to the corner of the glass. Align the PET sheet onto the glass. Using a paper towel begin slowly smoothing/applying the PET onto the glass while pulling back the clear sheet from underneath the PET. You can also spray glass cleaner onto the paper towel or the top side of the PET for a smoother application. After applying the PET sheet to the glass any bubbles can be pushed out or smoothed down using a credit card or stiff and dull piece of plastic.

\section{Hobbed Bolt}
\index{hobbed bolt}
\index{extruder jam}
The plastic filament is pulled through the extruder by a hobbed bolt. After repeated use, the teeth of the hobbed bolt can become filled with plastic. Using the brush or pick from the printer kit, clean out the hobbed bolt teeth. If an extruder jam ever occurs, remove the plastic filament from the extruder and clean out the hobbed bolt.

\section{Software}
\index{software}
\index{download}
Every quarter LulzBot will release a new stable version of the software. It is best to update the software every time a new version is released. The software is as important in printing quality parts as the hardware. Each quarterly software update can bring advances in print quality. The files are available at \texttt{http://download.lulzbot.com/TAZ/} . You can also find updated software versions in the Support/Downloads section of LulzBot.com.

\section{Belts}
\index{belts}
Over long periods or after extensive relocating of the printer you may need to re-tighten the belts on the TAZ 3D printer. For the X axis, using the 2.5mm hex driver, loosen one of the belt clamps. The belts clamps are located on the X axis carriage. To loosen the belt clamp, loosen the M3 screws on each side of the clamp. Using the needle nose pliers, pull the belt tight. While holding the belt tight, tighten down both M3 screws. The Y axis belt can be tightened in the same steps as the X axis with the belt clamps found on the bottom of the Y axis plate. Make sure not to over tighten the belts as this can cause unneeded stress on the printer.

\section{Hot End}
\index{hot end}
\index{acetone}
The hot end should be kept clean of extruded plastic by removing melted plastic strands with the tweezers. If melted plastic builds up on the hot end nozzle it can be cleaned with a paper towel soaked with acetone. Make sure the hot end is completely cool before attempting to clean the nozzle with acetone.


\section{Electronics}
\index{RAMBo}
\glossary{RAMBo}{[R]epRap [A)]duino-[M]ega compatible [M]other [Bo]ard. Designed by Joynnyr of Ultimachine.}
\index{electronics}
\index{fan}
The electronics case holding the RAMBo board may need to have dust blown out occasionally. Power down the printer and use the provided 2.5MM driver to remove the 4 M3 screws holding the lid to the enclosure. \textcolor{red}{The fan is mounted to the lid and connected to the RAMBo board. Be careful with the fan cable during removal}. Once removed use short bursts of compressed air to blow out any dust or debris. Plug in the lid fan paying attention to polarity and reattach the lid.
